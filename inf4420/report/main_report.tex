\documentclass[english, 12pt, a4paper]{ifimaster}
\usepackage[utf8]{inputenc} 
\usepackage[T1]{fontenc,url} %url
\urlstyle{sf} %sf
% \usepackage{lmodern}  % For times new roman : mathptmx
% \usepackage{mathptmx}
\usepackage[toc,page]{appendix}
\usepackage{babel,textcomp,csquotes,ifikompendiumforside, varioref,graphicx, gensymb}
% \usepackage[]{circuitikz}
% \usepackage{tikz}
\usepackage[backend=biber,style=numeric-comp, sortcites]{biblatex} 

% --------------------------------------------------
\title{ \huge{Successive Approximation Register (SAR) Analog-to-Digital Converter (ADC)}}
% \subtitle{Legg til en subtitle dersom det trengs}
\author{By Espen Klein Nilsen\\ and Vegard Midtbøen}
% --------------------------------------------------

\bibliography{ref} 


\begin{document}

\ififorside{}
\frontmatter{}
\maketitle{}
%-----------Acknowledgment---
\chapter*{Acknowledgment}

%-----------Abstract---------
\chapter*{Abstract}

\tableofcontents{}
\listoffigures{}
\listoftables{}

%-----------Preface----------
\chapter*{Preface}


%-----------Introduksjon-----
\mainmatter{}        

\chapter{Introduction}         
Critical parts in SAR ADCs 
\begin{itemize}
 \item Comparator
 \item DAC
\end{itemize}
SAR ADCs limitations:
\begin{itemize}
 \item The settling time is the DAC, whick must settle within the resolution of the overall converterm for example, 1/2 LSB (Least significant bit)
 \item The comparator, which must resolve small differences in $V_{in}$ and $V_{DAC}$ within the specified time
 \item The logic overhead
\end{itemize}
Comparator:
\begin{itemize}
 \item The comparator does not effect the overall linearity 
\end{itemize}




\section{Overview of the report}

\section{Summary}

%-----------Background-------
\chapter{Background}


%-----------Main part--------          
\chapter{Planning the project} 
\section{Task 1}
\subsection{Task 1.B}
\begin{table}[]
\centering
\caption{Comparation of different ADCs}
\label{comp:adc}
\begin{tabular}{|l|l|l|l|l|}
\hline
                 & SAR        & Flash   & Sigma-Delta  & Integrating                      \\ \hline
Conversion speed & Low-medium & Fast    & Low          & Depending on reolution and clock \\ \hline
Sampling rate    & Nyquist    & Nyquist & Oversampling & Nyquist                          \\ \hline
Area             & Low        & High    & Medium       &                                  \\ \hline
\end{tabular}
\end{table}



%-----------Results----------
\chapter{Results} 
\chapter{Discussion and conclusion}
\section{Discussion}
\subsection{Proposal of improvement}
\section{Conclusion}

\backmatter{}


\begin{appendices}
\chapter{Software and instrument overview}
\end{appendices}

\printbibliography{}

\end{document}


% --------------------------------------------------------------------
% Fotnote eksempel
% Hei \footnote{\url{http://www.google.fi/}}
% --------------------------------------------------------------------
% Figur eksempel
% \begin{figure}[!ht]
%   \centering
%     \includegraphics[width=0.5\textwidth]{filplassering}
%     \caption{Bildetekst}
%     \label{fig:}
% \end{figure}
% ---------------------------------------------------------------------
% For en mer bestemt plassering:
% 
% \begin{wrapfigure}{r}{0.5\textwidth}
%   \begin{center}
%     \includegraphics[width=0.48\textwidth]{filplassering}
%   \end{center}
%   \caption{Bildetekst}
%   \label{fig:}
% \end{wrapfigure}
% 
% Plasseringer for bilde:
% r 	R 	right side of the text
% l 	L 	left side of the text
% i 	I 	inside edge–near the binding (in a twoside document)
% o 	O 	outside edge–far from the binding
% ----------------------------------------------------------------------
% For å innkludere skjematiske tegninger:
%
%\begin{circuitikz}[scale=1.25]

% \draw (-1,0) node[anchor=east] {} to [short, *-*] (1,0);
% \draw (-1,2) node[anchor=east] {} to [inductor, *-*,  l=$\Delta x L$] (1,2);
% \draw (-1,0) to [open, l=$\cdots$] (-1,2);    
% \draw (3, 0) to (1, 0) to [capacitor, l=$\Delta x C$, *-*] (1, 2) to [inductor, *-*, l=$\Delta x L$] (3, 2);
% \draw (5, 0) to (3, 0) to [capacitor, l=$\Delta x C$, *-*] (3, 2) to [inductor, *-*, l=$\Delta x L$] (5, 2);
% \draw (7, 0) to (5, 0) to [capacitor, l=$\Delta x C$, *-*] (5, 2) to [inductor, *-*, l=$\Delta x L$] (7, 2);
% \draw (9, 0) to (7, 0) to [capacitor, l=$\Delta x C$, *-*] (7, 2) to [inductor, *-*, l=$\Delta x L$] (9, 2);
% \draw (9,0) node[anchor=east] {} to [short, *-*] (9,0);
% \draw (10,0) to [open, l=$\cdots$] (10,2);

% \end{circuitikz}
% ----------------------------------------------------------------------
% Paranteser
% 
% \big( \Big( \bigg( \Bigg( 	
% \big] \Big] \bigg] \Bigg] 	
% \big\{ \Big\{ \bigg\{ \Bigg\{ 	
% \big \langle \Big \langle \bigg \langle \Bigg \langle 	
% \big \rangle \Big \rangle \bigg \rangle \Bigg \rangle 	
% -----------------------------------------------------------------------
