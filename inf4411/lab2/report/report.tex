%\documentclass[a4paper,english,11pt,twoside]{article}
\documentclass[a4paper,english,11pt]{article}

\usepackage[utf8]{inputenc}
\usepackage[T1]{fontenc, url}
\usepackage[english]{babel}
%\usepackage{epsfig}
\usepackage{graphicx}
\usepackage{amsmath}
\usepackage{mathtools}
\usepackage{pstricks}
\usepackage{subfig}
\usepackage{epstopdf}
\usepackage{varioref}
\usepackage{listings}
\usepackage{xcolor}
\usepackage{float}
\usepackage[]{mcode}
\usepackage{verbatim}
\lstset{ 
  captionpos=b,
  frame=tb,
  numbers=left}
\urlstyle{sf}
\usepackage[margin=1 in]{geometry} % Setter margene til word standard

\usepackage{ifikompendiumforside}


\newcommand{\tab}[1]{\hspace{.2\textwidth}\rlap{#1}}

%%%%%%%%%%%%%%%%%% END HEADER

\title{Laboratory Assignment 2}
\subtitle{INF4411\\ 
          Analog Microelectronics}

\author{
\begin{tabular}{ r c l }
  Rikesh Chauhan & & rikesh.chauhan@fys.uio.no\\
  Espen Klein Nilsen & & e.a.k.nilsen@fys.uio.no\\
  Vegard Midtbøen & & vegard.midtboen@fys.uio.no
\end{tabular}
}
%{Rikesh Chauhan rikesh.chauhan@fys.uio.no\\
%	Espen Klein Nilsen e.a.k.nilsen@fys.uio.no\\
%	Vegard Midtbøen vegard.midtboen@fys.uio.no} 

\begin{document}
\ififorside
        
\section{Task 1}
In order to get a voltage reference to V$_{p_bias1}$, we used a potentiometer as seen in figure (figure reference).
For the task, we used a Vdd = 5V, which is enough for this task. 
This voltage is fixed trough the whole simulation. 
When we selected the V$_{p_bias1}$, we coupled up the circuit as shown in figure (figure ref). In the labdescription it was
gived that the bias current should be 20$\mu$A for the transistor to be in the active region.
V$_{p\_bias}$ = 1.196V\\
rds = 391k$\varOmega$\\
minimal V$_{out}$ = 1.818V\\
Headroom voltage is required to keep the device in active region MORE DETAILS!\\
    
\begin{figure}[htbp]
 \centering
  \fbox{\includegraphics[width=\textwidth]{img/Transistor_setup.jpg}}
  \caption{Transistor setup.}
  \label{fig:tran-setup}	
\end{figure}

\begin{figure}[htbp]
 \centering
  \fbox{\includegraphics[width=\textwidth]{img/task1_pmos_as_a_current_source.png}}
  \caption{pmos as a current source.}
  \label{fig:pmos-as-current-source}	
\end{figure}

\newpage
\section{Task 2}
Notes:\\
- Used two potentiometers to make the desirable voltage for V$_{p_bias1}$ and V$_{p_bias2}$.\\
- VDD (+25V port) = 5V\\
- V$_{sweep}$ = 2.5V\\
- V$_{p_bias1}$ = 1.196V (From last task)\\
- V$_{p_bias2}$ = 3.39V at 20 $\mu$ A\\
- Minimum voltage different is VDD - V$_{p\_bias2}$ = 5V - 3.39V = \underline{\underline{1.61V}}\\
- Minimun output voltage is 2.83V as seen in Figure \ref{fig:pfet-cascode-pfet}.\\
- Output resistance is 13.35M$\Omega$\\
\\
Figure \ref{fig:pfet-cascode-pfet} shows plot of output current as function of V$_{out}$.
\begin{figure}[htbp]
 \centering
  \fbox{\includegraphics[width=\textwidth]{img/task2_pmos_as_a_current_source.png}}
  \caption{pFET and cascode pFET as a current cource.}
  \label{fig:pfet-cascode-pfet}	
\end{figure}

\newpage
\section{Task 4}
Doing the same as we did in Task 1 and Task 2, but here we uses nmos transistors instead.\\
\\
1. First we maually find the new V$_n\_bias$ for the first nmos transistor at 20 uA.\\
Uses the same nmos transistor as we did in the lab 1. (port 6(G),7(S) and 8(D))\\
2. Then we use the voltage that we found in the first task and use it in the next task, whis is applying one more nmos transistor. Here we
use the ports 9 (S), 10 (G) and 12 (D). (Reference to picture of setup)


\end{document}
